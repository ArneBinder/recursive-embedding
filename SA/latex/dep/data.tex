\section{Data}\label{Sec:Data}

\begin{itemize}

    \item Describe the data and its quality.
    \item How was the data sample selected?
    \item Provide descriptive statistics such as:
        \begin{itemize}
            \item time period,
            \item number of observations, data frequency,
            \item mean, median,
            \item min, max, standard deviation,
            \item skewness, kurtosis, Jarque--Bera statistic,
            \item time series plots, histogram.
        \end{itemize}
    \item For example:
        \begin{table}[ht]

        \begin{center}
            {\footnotesize
            \begin{tabular}{l|cccccccccc}
                \hline \hline
                           & 3m    & 6m    & 1yr   & 2yr   & 3yr   & 5yr   & 7yr   & 10yr  & 12yr  & 15yr   \\
                \hline
                    Mean   & 3.138 & 3.191 & 3.307 & 3.544 & 3.756 & 4.093 & 4.354 & 4.621 & 4.741 & 4.878  \\
                    StD    & 0.915 & 0.919 & 0.935 & 0.910 & 0.876 & 0.825 & 0.803 & 0.776 & 0.768 & 0.762  \\
                \hline \hline
            \end{tabular}}
        \end{center}
        \caption{Some descriptive statistics of location and dispersion for
        2100 observed swap rates for the period from February 15, 1999
        to March 2, 2007. Swap rates measured as 3.12 (instead of 0.0312). See Table
        \ref{Tab:DescripStatsRawDataDetail} in the appendix for
        more details.}
        \label{Tab:DescripStatsRawData}
        \end{table}

    \item Allows the reader to judge whether the sample is biased or to evaluate possible impacts of outliers, for
    example.

\end{itemize}
