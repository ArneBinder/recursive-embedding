\section{Results}\label{Sec:Results}

\begin{itemize}

    \item Organize material and present results.

    \item Use tables, figures (but prefer visual presentation):
        \begin{itemize}
            \item Tables and figures should supplement (and not duplicate) the
                text.

            \item Tables and figures should be provided with
            legends.\\
                {\it Figure \ref{Fig:Resids} shows how to include and reference
                graphics. The graphic must be labelled before. Files must be in
                \texttt{.eps} format.}

                \begin{figure}[ht]
                \begin{center}
                    \includegraphics[scale=0.5,angle=0]{graph}
                    \caption{Estimated residuals from model XXX. ...}
                    \label{Fig:Resids}
                \end{center}
                \end{figure}

            \item Tables and graphics may appear in the text or in
                the appendix, especially if there are many simulation results
                tabulated, but is also depends on the study and number of tables resp.
                figures. The key graphs and tables must appear in
                the text!
        \end{itemize}

    \item Latex is really good at rendering formulas:\\
        {\it Equation (\ref{Eq:SpecDens}) represents the ACs of a stationary
        stochastic process:
        \begin{equation}
            f_y(\lambda) = (2\pi)^{-1} \sum_{j=-\infty}^{\infty}
                           \gamma_j e^{-i\lambda j}
                         =(2\pi)^{-1}\left(\gamma_0 + 2 \sum_{j=1}^{\infty}
        \gamma_j \cos(\lambda j)\right)
                                        \label{Eq:SpecDens}
        \end{equation}
        where $i=\sqrt{-1}$ is the imaginary unit, $\lambda \in [-\pi,
        \pi]$ is the frequency and the $\gamma_j$ are the autocovariances
        of $y_t$.}

\newpage

    \item Discuss results:
        \begin{itemize}
            \item Do the results support or do they contradict economic theory ?
            \item What does the reader learn from the results?
            \item Try to give an intuition for your results.
            \item Provide robustness checks.
            \item Compare to previous research.
        \end{itemize}
\end{itemize}
